\documentclass[11pt,a5paper,twoside]{memoir}
%%%%%%%%%%%%%%%%%%%%%%%%%%%%%%%%%%%%%%%%%%%%%%%%%%%%%%%%%%%%%%%%%%%%%%%%
\usepackage[svgnames]{xcolor}
\usepackage{pst-barcode}
\usepackage{auto-pst-pdf}
\usepackage{ifthen}
\usepackage{everypage}
\usepackage{enumitem}
\usepackage{DejaVuSerif}
\usepackage{DejaVuSans}
%%%%%%%%%%%%%%%%%%%%%%%%%%%%%%%%%%%%%%%%%%%%%%%%%%%%%%%%%%%%%%%%%%%%%%%%
\chapterstyle{bianchi}
\pagestyle{empty}

\colorlet{pagenumbg}{black!70!white}
\colorlet{pagenumfg}{white}

\makeatletter
\AddEverypageHook{%
  \ifthenelse{\isodd{\value{page}}}%
	{\rput[tl](\textwidth,0){%
		\psframebox[fillstyle=solid,fillcolor=pagenumbg,linestyle=solid,linecolor=white,framesep=6pt,linearc=.5]{\color{pagenumfg}\Large\bfseries\sffamily\thepage\hspace*{2cm}}}%
  }
	{\rput[tr](0,0){%
		\psframebox[fillstyle=solid,fillcolor=pagenumbg,linestyle=solid,linecolor=white,framesep=6pt,linearc=.5]{\color{pagenumfg}\Large\bfseries\sffamily\hspace*{2cm}\thepage}}%
	}
}
\makeatother

%%%%%%%%%%%%%%%%%%%%%%%%%%%%%%%%%%%%%%%%%%%%%%%%%%%%%%%%%%%%%%%%%%%%%%%%
\begin{document}

% London Hackspace
% Laboratory 24
% 37 Cremer Street, E2 8HD
% https://london.hackspace.org.uk

\chapter{Welcome to London Hackspace}
\colorlet{pagenumbg}{red!80!black}

This is the largest hackspace in the UK, and you are free (and encouraged) to tinker and create using our tools and expertise.

To do this, we need your help as well. Your membership fee is a big part of this, and helps us pay our rent and utility bills. We also need you to keep the space running and looking as you found it. If you can improve our facilities, bring cake (and clear up the crumbs!), or even just help keep it tidy, so much the better.

In-person contact with awesome people is one the best things about the Hackspace, but we have far more members online than are physically in the space at any time, so we strongly encourage you to participate online as well. We use IRC and the london-hackspace mailing list (check the website for details) for technical talk, and for the organisation of the space.

If you haven't been to the space yet, Tuesday evenings are the best time for newcomers to visit, and there will be tours of the space. Also ask everyone what they're working on - people love to talk at length about their projects.

Some of our equipment is dangerous or easily damaged, so we need to restrict its use (eg. the 3-in-1 mill/drill/lathe, and the laser cutter). These will be clearly labelled, and often secured with a key. Ask on the mailing list, and someone will usually offer to show you the ropes \& give you access. Don't attempt to use them without training, because we hate clearing up blood \& body parts even more than we hate waiting six weeks for new tools to arrive.

\section{Access}

You can open the door to London Hackspace 24 hours a day using an RFID card (such as your current Oyster card), if you're a member - the wiki has instructions on setting this up.

\subsection{Out of Hours access}

The main gates are sometimes shut with the chain draped over them, and sometimes completely locked at night. If they're locked or you have any problems, ask the security guard who will be in his office next to the gates. There is a keypad lock on the stairwells, which restricts access between 7pm and 7am. If you are a member, you can find the code on the members page. If not, contact someone in the Hackspace by IRC or phone to be let in.

\subsection{Car parking}

We have two parking slots, which you can use with a permit - the permits are kept inside the space, attached to large pieces of wood. Please return them when you leave!

\subsection{Bike storage}

There are a few places to store bikes:

\begin{itemize}
	\item Cycle rack in the car park (use a good lock, not recommended at night as we have had theft attempts)
	\item On the main balcony in front of the space (you need a sizeable lock to lock around the railings)
	\item Please don't keep your bike in the space unless you're working on it.
\end{itemize}

There is a Barclays Cycle Hire station in Front of the entrance to Hoxton Station, there are also two other docking stations nearby:

\begin{enumerate}
	\item Opposite Hackney Community College, Falkirk Street.
	\item Shoreditch High Street, near the church.
\end{enumerate}


\section{Tuesday social evenings}

Where people hang out and hack on projects, socialise and collaborate. Bring beer if you like. Every Tuesday at 7:00pm-ish til whenever people go home, generally very late. Everyone is welcome, both members and non-members. It tends to be a nice mix of regulars, irregulars and new faces.

\section{Why?}

\begin{itemize}
	\item Simply to take a look around, see the MakerBot, Laser Cutter, etc. and have a chat.
	\item Because you want to hack on projects but don't get time.
	\item Because you need help with something.
	\item Because you want to see the space and play with our gear before becoming a member.
	\item Because we have tools you need to use that you don't have at home.
	\item Because you just want to hang out with like minded people.
	\item Because you want to get involved with the organisation of the London Hackspace.
\end{itemize}

\section{Rules}

London Hackspace is run entirely by its members and, on the whole, this works pretty well. As hackers we hate making rules almost as much as we hate following them, so we really want to keep the number of rules to a minimum. We can only do this if members and visitors observe the spirit -- not just the letter -- of these rules; they are here firstly for your safety, and secondly to prevent annoyance to other users of the space.

\subsection{Rule Zero}

\begin{enumerate}[start=0]
	\item Do not be on fire.
\end{enumerate}

\subsection{Safety}

\begin{enumerate}[resume]
	\item Don't use power tools unless you've had a safety briefing on them first.
	\item Don't defeat or hack safety features/equipment. This is for other people's safety as much as yours.
	\item Check the wiki for instructions. Read the warnings. If in doubt, ask.
\end{enumerate}

\subsection{Making Decisions}

\begin{enumerate}
	\item If something is broken, fix it; don't complain.
	\item If you're doing something major, ask the mailing list first.
\end{enumerate}

\subsection{Etiquette}

\begin{enumerate}
	\item Do not treat the Hackspace like your home, it is a shared space. Sleeping in the space is forbidden.
\end{enumerate}

\subsection{Personal Items}

\begin{enumerate}
	\item Members are allowed to store personal items in the space, but they must be kept in one of the plastic boxes, labelled with your name. One box per member. Empty boxes may be recycled for other members if needed.
	\item Larger/more items may be allowed, but please note that space is limited. You must email the mailing list first, and items should be clearly labelled as yours, and with the date they will be removed by. (We have Do Not Hack stickers in the entrance area. If you can't find them, ask someone.)
\end{enumerate}

\subsection{Donating, Loaning or Borrowing Items}

\begin{enumerate}
	\item Carefully consider the true usefulness of an item before bringing it to the space. Disposal requires effort and space is limited. Seek permission for large items.
	\item If you're leaving something and you would rather it was not mercilessly ripped apart, please label it accordingly.
	\item Likewise, if something looks expensive or useful please don't mercilessly rip it apart without asking first.
	\item Don't remove tools from the space without asking the mailing list first.
\end{enumerate}

\subsection{Tidiness}

\begin{enumerate}
	\item Workbenches should be completely clean when you leave - be considerate for the next user. Please put tools back where you found them.
	\item Put your dirty cups and plates in the dishwasher before leaving the space. Don't leave them in the sink.
	\item Any items left on a workbench overnight become fair game. Put your stuff back in your box.
	\item If you have engaged in any waste generating activity (e.g. peeling wires), hoover the spot.
	\item Please do not bring bikes into the space unless you are working on them, they take up too much room.
\end{enumerate}

\subsection{Disposal}

\begin{enumerate}
	\item If you think something should be thrown out, put it in the outgoing trash box. Two weeks later the box will go in the skip.
	\item If something is too big to fit in the trash box, you should probably post to the list about it.
	\item If you want to throw something away which looks useful, make sure the mailing list is informed well in advance.
\end{enumerate}

\subsection{Penalties}

If you continually annoy or endanger others by ignoring these rules, the trustees have the right to impose sanctions up to stripping you of membership and banning you from the space. It is not something they want to do.

\section{Governance}

London Hackspace is run by its members. If you have an opinion or want to raise an issue, then please do - as a subscribing member your voice is as important as anyone else's.

London Hackspace Ltd.'s ultimate governance lies with the Board of Trustees, who are also the directors of the company. These are elected regularly, with the longest-serving third of the Board required to stand for re-election every year.

\subsection{Current Trustees}

\begin{tabular}{lll}
	\textbf{Name}         & \textbf{Username} & \textbf{Appointed} \\
	Russ Garrett          & russss    & 2010-07-27* \\
	Charles Yarnold       & solexious & 2010-07-27 \\
	Robert Leverington    & roberthl  & 2010-07-27 \\
	Andy "Bob" Brockhurst & b3cft     & 2010-07-27 \\
	Jonty Wareing         & jontyw    & 2011-08-06* \\
	Martin Dittus         & Martind   & 2011-08-06 \\
	Mark Steward          & ms7821    & 2011-08-06 \\
	Philip Roy            & cepmender & 2011-08-06 \\
\end{tabular}

* re-appointed

\subsection{Directors' Responsibilities}

All UK companies must have have a board of directors, but directorship of London Hackspace Ltd. is primarily a symbolic role. We want trustees to simply make sure that the company is doing what its members want.

In order to do that, there are regular "talk to a Trustee" sessions where you can raise any issues, or ask for help, in confidence. Please take advantage of them!


\section{Events}

Google Calendar (see the wiki for details) is our canonical list of events, but here's a selection. All events listed are at the space itself unless otherwise noted.

\subsection{Recurring}

\begin{itemize}
	\item Every Monday, 19:30 - Gadget Geeks viewing in quiet room
	\item Every Tuesday, 19:00 - Weekly public meetings
	\item Every Wednesday, 19:30 - Biohacking catch-up
	\item Every Wednesday, 19:30 - Mind Hackers
	\item Every Thursday, 19:00 - Music Hack Space meetings
	\item 3rd Thursday, 19:30 - OneClickOrgs hack evening
	\item 1st Saturday, 13:00 - OneClickOrgs hack day
	\item 2nd Saturday, 15:00 - Lockpicking sports session
\end{itemize}



\chapter{Keeping the Hackspace in Shape}
\colorlet{pagenumbg}{blue!85!black}

According to our rules you're expected to clean up your workspace after you're done. Beyond that we ask that you help us maintain the space. We have a lot of members that make constant use of our facilities, and it adds up quickly. A few good rules of thumb:

\begin{itemize}
	\item Strive to do one maintenance task every time you visit. You're already helping if you only bring out the garbage or wipe a few desks clean.
	\item Strive to leave our facilities in a better state than you found them.
	\item If you don't know how things work and the wiki doesn't help then just ask around, in the space or on IRC. Then update the wiki for the next person.
\end{itemize}

\subsection{Waste Disposal}

\subsubsection{Regular Waste}

This includes industrial waste (electronics, metal, stone, wood, ...) as long as it is not toxic. We have set up a few bins in various places. Once they're full take out the bag, replace it with a new one (you can find them under the kitchen sink), and dispose of the full bag. There are two black bins in the Cremer Business Centre parking space outside our house, to your left when you enter the main gate. There are two more half-way down the carpark.

\subsubsection{Recycling}

We've acquired some bright orange skips in the business centre - these are marked as recycling skips. There are also fewer normal skips than before so recycling is now encouraged. As far as we can tell, we can recycle:

\begin{itemize}
	\item Plastics (recyclable ones)
	\item Cans \& foil
	\item Glass bottles
	\item Cardboard \& paper
\end{itemize}

There should be a bin labelled in the space for recycling - please use it. Also, please empty it out regularly.

\subsection{Cleaning}

Most cleaning utensils can be found under the kitchen sink. (See also the "re-stocking supplies" section further below.) We have a small dishwasher next to the kitchen sink., and several vacuum cleaners which normally live in the workshop. Supplies are currently stored under the kitchen sink, and above the black fridge. If we run out of anything (e.g. kitchen gloves go quite quickly) it would be appreciated if you bought more. The Hackspace can reimburse you if needed - talk to a trustee about that.

\chapter{Quick-reference QR codes}
\colorlet{pagenumbg}{red!60!yellow!80!black}


The main hackspace page

The wiki

How to get to the Space

The mailing list

Announcements mailing list

 Infrastructure mailing list

wifi

VCard with all the space contact details







	\begin{pspicture}(1in,1in)
		\psbarcode{WIFI:T:WPA;S:LondonHackspace;P:DFDE595F79;;}{eclevel=M width=1.0 height=1.0}{qrcode}
	\end{pspicture}

	\vspace*{12pt}

	\begin{pspicture}(2in,2in)
	\psbarcode{MECARD:\
N:London Hackspace;\
TEL:+442076135391;\
ADR:Laboratory 24, Unit 24, 37 Cremer Street, London, E2 8HD, GB;\
URL:https://london.hackspace.org.uk/;\
URL:http://wiki.london.hackspace.org.uk/view/London_Hackspace;;}{eclevel=M width=2.0 height=2.0}{qrcode}
	\end{pspicture}

	\vspace*{12pt}

	\begin{pspicture}(1in,1in)
		\psbarcode{geo:51.53034,-0.07660?q=51.53034,-0.07660(London+Hackspace)}{eclevel=M width=1.0 height=1.0}{qrcode}
	\end{pspicture}

\end{document}
